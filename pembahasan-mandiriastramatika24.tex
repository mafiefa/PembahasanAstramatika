\chapter{Pembahasan Soal Mandiri}
\section{Astramatika XXIV}

\setenumerate{itemsep=2pt, topsep=2pt}
\begin{enumerate}

\item \textbf{Jawaban: -6.} Kita lihat dari kata \emph{\sffamily ``Jika ${\mathsf{x}^\mathsf{2}+\mathsf{2}\mathsf{x}=\mathsf{2}\mathsf{4}}$ \ldots''}, didapat \[x^2+2x-24 = 0\Leftrightarrow (x+6)(x-4) = 0\]nilai $x$ yang memenuhi adalah $x=-6$ dan $x=4$. Kemudian kita substitusikan ke pertidaksamaan pada kata \emph{\sffamily``\ldots maka ${\mathsf{x}^\mathsf{2}+\mathsf{4}\mathsf{x}>\mathsf{1}\mathsf{7}}$''}, saat $x = -6$ didapat \[x^2 + 4x = (-6)^2 + 4(-6) = 36 - 24 = 12 < 17\]kemudian saat $x=4$ didapat \[x^2+4x = (4)^2 + 4(4) = 32 > 17\]maka nilai $x$ yang membuat pernyataan pada soal bernilai salah adalah $x=-6$.

\item \textbf{Jawaban: 32.} Karena $p,q$ adalah akar-akar persamaan $x^2-3x+4=0,$ maka \[p^2-3p+4 =0\Leftrightarrow p^2-p+4 = 2p\]kemudian\[q^2-3q+4 = 0\Leftrightarrow q^2+q+4=4q\]lalu dari formula vieta, kita tahu bahwa $pq = 4$, sehingga didapat \[(p^2-p+4)(q^2-3q+4) = (2p)(4q) = 8pq = 32\]

\item \textbf{Jawaban: 5.} Perhatikan bentuk $ax^2-3x+5$, bentuk persamaan kuadrat tersebut memiliki nilai diskriminan \[D = b^2-4ac = 9-20a < 0\text{ untuk $1<a<2$}\]yang membuat bentuk kuadrat $ax^2-3x+5$ definit positif. Maka agar pertidaksamaan pada soal dapat terpenuhi, haruslah \begin{equation}\label{nomor3.1}-3x^2+8x+3 = (3x+1)(3-x) < 0\end{equation}bisa dicek bahwa solusi dari \eqref{nomor3.1} adalah $x<-1/3$ dan $x>3$. Maka $(p,q) = (-1/3, 3)$, sehingga nilai dari $3p +2q = 3\left(-\dfrac{1}{3}\right) + 2(3) = 5.$

\item \textbf{Jawaban: 8.} Titik balik fungsi kuadrat adalah titik $x$ yang memenuhi $f'(x) = 0$ dengan $f'(x)$ menyatakan turunan pertama fungsi $f(x)$. Oleh karena itu, berarti $x=-a$ memenuhi $f'(-a) = 0$, sehingga dengan \[f'(x) = 2ax + (a+6)\]didapat\[f'(-a) = 2a(-a)+(a+6) = -2a^2+a+6 = (2a+3)(2-a) = 0\]karena $a$ bilangan bulat, maka $a=2$. Berarti \[f(x) = 2x^2+8x+8 = 2(x+2)^2\]kemudian karena $0\leq (x+2)^2 \leq 4$, maka \[0\leq f(x)\leq 8\] yang berarti nilai maksimum $f(x)$ adalah 8, yang terjadi saat $x=0$.

\item \textbf{Jawaban: 2.} Karena garis $y=2x+k$ dan parabola $y=x^2-x+3$ saling berpotongan, maka \begin{equation}\label{nomor5.1}2x+k = x^2-x+3 \Leftrightarrow x^2-3x+(3-k) = 0.\end{equation} \par Dari soal kita tahu bahwa $a^2 + c^2 = 7$, yang berarti \[a^2 + c^2 = (a+c)^2 - 2ac = 7.\]\par Sekarang, karena $a,c$ memenuhi \eqref{nomor5.1}, maka $a+c = 3$ dan $ac = 3-k$, sehingga didapat \[(a+c)^2-2ac = 9-6+2k = 3+2k = 7\Leftrightarrow k = 2.\]

\item \textbf{Jawaban: -1/3.} Perhatikan bahwa \[\sqrt{\frac{b}{a}\sqrt{\frac{a}{b}\sqrt{\frac{b}{a}\sqrt{\frac{a}{b}\cdots}}}} = \sqrt{\frac{b}{a}\sqrt{\frac{a}{b}\cdot \frac{a^x}{b^x}}} = \sqrt{\frac{b}{a}\sqrt{\frac{a^{x+1}}{b^{x+1}}}} = \frac{a^x}{b^x}\]kemudian\[\sqrt{\frac{a^{x+1}}{b^{x+1}}} = \frac{a^{2x+1}}{b^{2x+1}}\Leftrightarrow \frac{a^{x+1}}{b^{x+1}} = \frac{a^{4x+2}}{b^{4x+2}}\]kemudian didapat \[\frac{a^{3x+1}}{b^{3x+1}} = 1\]yang hanya dapat terjadi jika $3x+1 = 0$ karena $a\neq b$, maka $x=-1/3$.

\item \textbf{Jawaban: ${\mathbf \pm}$1.} Perhatikan bahwa \[(a^x - a^{-x})^2 = (a^{2x} + a^{-2x}) - 2 = 3-2 = 1\]maka didapat $a^x - a^{-x} = \pm1.$

\item \textbf{Jawaban: -1.} Perhatikan bahwa \[\log_2 56 = \log_2 (2^3)(7) = 3+\log_2 7 = 7a\]kemudian karena $\log_7 2 = b\Leftrightarrow \log_2 7 = 1/b$, maka didapat \begin{equation}\label{nomor8.1}3+\frac{1}{b} = 7a.\end{equation}Kemudian karena $c = \log_4 1024 = 5$, maka \[\left(\frac{1}{b} + \frac{50}{c}\right)^2 = \left(\frac{1}{b} + 10\right)^2 = d = 0\]yang berarti $1/b = -10.$ Substitusikan $1/b = -10$ ke \eqref{nomor8.1} didapat \[3 - 10 = -7 = 7a\Leftrightarrow a = -1.\]

\item \textbf{Jawaban: 65.} Misalkan dua bilangan tersebut adalah $x,y$ dengan $x>y$. Maka
	\begin{gather}
	x-y = 5\label{nomor9.1}\\
	x^2+y^2 = (x+y)^2 - 2100\label{nomor9.2}
	\end{gather}
dari \eqref{nomor9.2} didapat $xy = 1050$, kemudian \[x^2+y^2 = (x-y)^2+2xy = (5)^2 + 2(1050) = 2125\]sehingga didapat \[(x+y)^2 = (x^2+y^2) + 2xy = (2125)+2(1050) = 4225\Leftrightarrow  x+y = 65\]

\item \textbf{Jawaban: 17.} Dari soal diketahui
	\begin{gather}
	3b+3x = 2a\label{nomor10.1}\\
	2x+a = b\label{nomor10.2}\\
	4a-5b = c\label{nomor10.3}
	\end{gather}
substitusi \eqref{nomor10.2} ke \eqref{nomor10.1}, didapat \[3(2x+a) + 3x= 9x + 3a = 2a\Leftrightarrow a = -9x \geq 9\text{ karena $x\leq -1.$}\]Selanjutnya, dari \eqref{nomor10.2} didapat $a = b-2x$, subsitusikan ke \eqref{nomor10.1} didapat \[3b+3x = 2(b-2x) = 2b-4x\Leftrightarrow b = -7x \geq 7\text{ karena $x\leq-1.$}\]Terakhir, subsitusikan $a=-9x$ dan $b=-7x$ ke \eqref{nomor10.3} sehingga didapat \[4a-5b = 4(-9x) - 5(-7x) = 35x - 36x = -x = c\geq 1\text{ karena $x\leq -1$,}\]sehingga nilai minimum dari $a+b+c$ adalah \[a+b+c \geq 9 + 7 + 1 = 17.\]

\item \textbf{Jawaban: 5/2.} Tiga titik $(x_1,y_1),(x_2,y_2),$ dan $(x_3,y_3)$ segaris jika \[\frac{x_2 - x_1}{y_2 - y_1} = \frac{x_3 - x_2}{y_3 - y_2}\]sehingga, karena titik $(3,2),(11,0),$ dan $(1,k)$ segaris, maka haruslah \[\frac{11-3}{0-2} = \frac{1-11}{k}\Leftrightarrow -4 = \frac{-10}{k}\Leftrightarrow k = \frac{5}{2}.\]

\item \textbf{Jawaban: 5.} Daerah yang diarsir pada gambar dibawah menunjukkan daerah penyelesaian dari sistem pertidaksamaan pada soal.
	\begin{figure}[H]
		\centering
		\begin{tikzpicture}[scale = 0.5]
			\fill[pattern color = red!40, pattern = north west lines] (0,14)--(12,14)--(12,0)--(10,0)--(4,6)--(0,12)--cycle;
			
			\draw[->] (0,0) -- (12,0)node[anchor = north]{$x$}; % x axis
			\draw[->] (0,0) -- (0,14)node[anchor = east]{$y$}; % y axis
			
			\node[draw, circle, fill = black, inner sep = 1.5] at (10,0){};
			\node[draw, circle, fill = black, inner sep = 1.5] at (0,10){};
			
			\node[anchor = north, yshift = -3] at (10,0){10};
			\node[anchor = east, xshift = -3] at (0,10){10};
			\node[anchor = north east] at (0,0) {$O$};
			
			\node[draw, circle, fill = black, inner sep = 1.5] at (8,0){};
			\node[draw, circle, fill = black, inner sep = 1.5] at (0,12){};
			
			\node[anchor = north, yshift = -3] at (8,0){8};
			\node[anchor = east, xshift = -3] at (0,12){12};
			
			\node[draw, circle, fill = black, inner sep = 1.5] at (4,6){};
			\node[anchor = south west] at (4,6){$P(4,6)$};
			
			\draw (10,0)--(0,10){};
			\draw (8,0)--(0,12){};
		\end{tikzpicture}
	\end{figure}
karena fungsi $f(x,y) = ax + 4y$ berada pada daerah yang diarsir, atau daerah penyelesaian sistem pertidaksamaan pada soal. Maka nilai $f(x,y)$ akan minimum pada salah satu titik $(10,0), (0,12),$ atau $(4,6)$. Perhatikan tabel dibawah
	\begin{table}[H]
		\centering
		\begin{tabular}{@{}lr@{}}
		\toprule
		$(x,y)$ & $f(x,y)$\\
		\midrule
		(10,0) & $10a$\\
		(0,12) & 48\\
		(4,6)&$4a+24$\\
		\bottomrule
		\end{tabular}
	\end{table}
karena $f(x,y)$ minimum saat $(x,y) = (4,6)$, maka haruslah \[f(10,0) = 10a > f(4,6) = 4a + 24\Leftrightarrow a > 4\] dan juga \[f(0,12) = 48 > f(4,6) = 4a + 24\Leftrightarrow a < 6\]sehingga haruslah $a=5$.

\item \textbf{Jawaban: $\boldsymbol{\{x\in\mathbb{R}\mid x>2\}}$.} Syarat pertama agar fungsi $f(x)$ terdefinisi adalah $x-2 \neq 0$, sehingga $x\neq 2$. Kemudian syarat yang lain adalah, \begin{equation}\frac{x^2-x+2}{x-2} \geq 0,\end{equation}selanjutnya perhatikan bahwa diskriminan dari $x^2-x+2$ adalah $$D = b^2-4ac = 1 - 8 = -7 < 0$$yang berarti $x^2-x+2$ definit positif. Maka agar (III.1) dapat terpenuhi, haruslah $x-2 > 0$ atau $x>2$. Maka daerah definisi fungsi $f(x)$ adalah $\{x\in\mathbb{R}\mid x>2\}$.

\item \textbf{Jawaban: -30.} Dari $f^{-1}(8x-10) = 6x - 2$, kita substitusikan $x\mapsto \dfrac{1}{8}x + \dfrac{5}{4}$ sehingga \begin{equation}\label{nomor14.1}\begin{split}f^{-1}(x) &= f^{-1}\left(8\left(\frac{1}{8}x + \frac{5}{4}\right)-10\right)\\&= 6\left(\frac{1}{8}x + \frac{5}{4}\right)-2 = \frac{3}{4}x + \frac{11}{2}.\end{split}\end{equation}Kemudian dari definisi invers fungsi $f(x)$, yaitu $f^{-1}(f(x)) = x$. Kita dapat mencari fungsi $f(x)$ sebagai berikut, substitusikan $x\mapsto f(x)$ ke \eqref{nomor14.1} didapat \[f^{-1}(f(x)) = x = \frac{3}{4}f(x) + \frac{11}{2}\Leftrightarrow f(x) = \frac{4x-22}{3}\]sehingga\[f(10) = \frac{4(10)-22}{3} = 6\]yang membuat \[(f^{-1}\circ f)(10) = f^{-1}(6) = \frac{3}{4}(6) + \frac{11}{2} = 10\]sehingga \[p^2-4p-20 = 10\Leftrightarrow p^2-4p-30 = 0.\]Maka berdasarkan formula vieta, hasil kali akar-akarnya adalah -30.

\item \textbf{Jawaban: 3/2.} Ingat bahwa \[\sin(360\degree + x) = \sin x\quad\text{dan}\quad\tan(180\degree +x) = \tan x\]kemudian karena\[2550\degree = 360\degree\times7 + 30\degree\quad\text{dan}\quad2020\degree = 180\degree\times11 + 45\degree\]maka\[\sin(2550\degree) + \tan(2025\degree) = \sin(30\degree) + \tan(45\degree) = \frac{1}{2} + 1 = \frac{3}{2}\]

\item \textbf{Jawaban: 24/25.} Perhatikan bahwa \[\cos\left(\frac{1}{2}\theta\right) = \sqrt{1-\left(\frac{\sqrt{5}}{5}\right)^2} = \frac{2}{5}\sqrt{5}.\] Dengan menggunakan \[\sin(2x) = 2\sin(x)\cos(x)\] beberapa kali, didapat \[\sin\left(2\times\frac{1}{2}\theta\right) = \sin(\theta) =2\sin\left(\frac{1}{2}\theta\right)\cos\left(\frac{1}{2}\theta\right) = 2\times\frac{1}{5}\sqrt{5}\times\frac{2}{5}\sqrt{5} = \frac{4}{5}\]kemudian, dari $\sin(\theta) = 4/5$ didapat \[\cos(\theta) = \sqrt{1-\left(\frac{4}{5}\right)^2} = \frac{3}{5}\]sehingga \[\sin(2\theta) = 2\sin(\theta)\cos(\theta) = 2\times\frac{4}{5}\times\frac{3}{5} = \frac{24}{25}.\]

\item \textbf{Jawaban: 0.} Ingat bahwa \[\cos(x) + \cos(y) = 2\cos\left(\frac{x+y}{2}\right)\cos\left(\frac{x-y}{2}\right)\]sehingga soal dapat diubah menjadi \begin{equation}\label{nomor17.1}\left(\cos\left(\frac{\pi}{8}\right) + \cos\left(\frac{5\pi}{8}\right)\right) + \left(\cos\left(\frac{3\pi}{8}\right)+\cos\left(\frac{7\pi}{8}\right)\right)\end{equation}kemudian \eqref{nomor17.1} menjadi
	\[
	2\cos\left(\frac{\pi}{4}\right)\left(\cos\left(\frac{3\pi}{8}\right)+\cos\left(\frac{5\pi}{8}\right)\right)=2\cos\left(\frac{\pi}{4}\right)\left(2\cos\left(\frac{\pi}{2}\right)	\cos\left(\frac{\pi}{8}\right)\right).
	\]
Karena ada faktor $\cos\left(\pi/2\right)$ di sebelah kanan yang sama dengan nol, maka \eqref{nomor17.1} sama dengan nol.

\item \textbf{Jawaban: 1/4.} Perhatikan bahwa
\begin{align*}
\frac{(2x-3\sqrt{x} +1)(\sqrt{x}-1)}{(x-1)^2} &= \frac{(2\sqrt{x}-1)(\sqrt{x}-1)(\sqrt{x}-1)}{\left[(\sqrt{x}-1)(\sqrt{x}+1)\right]^2}\\
							 &= \frac{(2\sqrt{x}-1)(\sqrt{x}-1)^2}{(\sqrt{x}-1)^2(\sqrt{x}+1)^2}\\
							 &= \frac{2\sqrt{x}-1}{(\sqrt{x}+1)^2}
\end{align*}
sehingga \[\lim_{x\to1}\frac{(2x-3\sqrt{x} +1)(\sqrt{x}-1)}{(x-1)^2} = \lim_{x\to1}\frac{2\sqrt{x}-1}{(\sqrt{x}+1)^2} = \frac{2(1)-1}{((1)+1)^2} = \frac{1}{4}\]

\item \textbf{Jawaban: $\boldsymbol{4(\sin x+\cos x)^3(\cos x-\sin x)}$.} Misalkan \[\sin x + \cos x = g(x)\implies g'(x) = \cos x - \sin x,\]maka dengan aturan rantai \[f(x) = [g(x)]^4 \implies f'(x) = 4[g(x)]^3g'(x)\]sehingga\[f'(x) = 4(\sin x +\cos x)^3(\cos x - \sin x)\]

\item \textbf{Jawaban: 0.} Misalkan $f'(x)$ menyatakan turunan pertama dari $f(x)$. Kemudian misalkan $$f_n(x) = \underbrace{(f_1\circ f_1\circ\cdots\circ f_1)}_{\text{$f_1$ muncul $n$ kali}}(x)$$dengan $f_1(x) = -\cos x$, contohnya $$f_2(x) = -\cos(-\cos(x))\quad\text{dan}\quad f_3(x) = -\cos(-\cos(-\cos(x))).$$ Dengan definisi diatas, soal dapat dinyatakan dengan $$L = \biggl\lvert\lim_{n\to\infty}f'_n(x)\biggr\rvert_{x=\frac{\pi}{2}}.$$Perhatikan bahwa dengan aturan rantai $$f'_2(x) = \frac{\text{d}f_2(x)}{\text{d}f_1(x)}\cdot\frac{\text{d}f_1(x)}{\text{d}x}\quad\text{dan}\quad f'_3(x) = \frac{\text{d}f_3(x)}{\text{d}f_2(x)}\cdot\frac{\text{d}f_2(x)}{\text{d}f_1(x)}\cdot\frac{\text{d}f_1(x)}{\text{d}x},$$sehingga $$L = \biggl\lvert\lim_{n\to\infty}\frac{\text{d}f_n(x)}{\text{d}f_{n-1}(x)}\cdot\frac{\text{d}f_{n-1}(x)}{\text{d}f_{n-2}(x)}\cdots\frac{\text{d}f_2(x)}{\text{d}f_1(x)}\cdot\frac{\text{d}f_1(x)}{\text{d}x}\biggr\rvert_{x=\frac{\pi}{2}}.$$Dari bentuk diatas, jika kita tinjau \begin{equation}\label{nomor20.1}\frac{\text{d}f_2(x)}{\text{d}f_1(x)} = \frac{\text{d}}{\text{d}(-\cos x)}(-\cos(-\cos x)) = \sin(-\cos x)\end{equation}saat $x={\pi}/{2}$, ekspresi \eqref{nomor20.1} bernilai \[\sin\left(-\cos\frac{\pi}{2}\right) = 0\]sehingga didapat $L = 0$

\item \textbf{Jawaban: $\boldsymbol{\dfrac{1}{15}}\boldsymbol{\sin^5(3x) + C}$.} Misalkan \[\sin(3x) = u\implies 3\cos(3x)\dx = \du\]sehingga\[\int \sin^4(3x)\cos(3x)\dx = \frac{1}{3}\int u^4\du = \frac{1}{15}u^5 + C = \frac{1}{15}\sin^5(3x)+C\]untuk suatu $C\in\mathbb{R}$.

\item \textbf{Jawaban: 18$\boldsymbol{\pi}$.} Titik potong $y = 2x$ dan $x=3$ adalah $(x,y) = (3,6)$. Sehingga jika diputar mengelilingi garis $x=3$, volumenya adalah \[V = \pi\int_{0}^{6} \left(\frac{y}{2}\right)^2\dy = \pi\left[\frac{1}{12}y^3\right]_0^6 = \frac{216\pi}{12} =18\pi\]

\item \textbf{Jawaban: $\boldsymbol{90\degree}$.} Misalkan $\vec{A} = 3\hat{\imath} + 4\hat{\jmath} + 2\hat{k}$ dan $\vec{B} = 2\hat{\imath} - 3\hat{\jmath} + 3\hat{k}$. Ingat bahwa \[\vec{A}\cdot\vec{B} = \lvert A\rvert\lvert B\rvert\cos{\theta}\]dengan $\theta$ menyatakan sudut antara vektor $\vec{A}$ dan $\vec{B}$ yang berarti \[\cos\theta = \frac{\vec{A}\cdot\vec{B}}{\lvert A\rvert\lvert B\rvert}.\]Perhatikan bahwa \[\vec{A} \cdot \vec{B} = 6 + (-12) + 6 = 0\]sehingga $\cos\theta = 0$ atau $\theta = 90\degree$.

\item \textbf{Jawaban: -2.} Misalkan \[f(x) = ax^{2016} - bx^{2017} - 2x + 1,\]perhatikan bahwa $f(x)$ dapat dinyatakan sebagai berikut \begin{equation}\label{nomor24.1}f(x) = (x^2-1)P(x) + (x+2)\end{equation} untuk suatu polinomial $P(x)$. Kemudian perhatikan bahwa dari \eqref{nomor24.1} didapat \[f(-1) = 1\]kemudian nilai dari $a+b$ dapat kita cari sebagai berikut \[f(-1) = a(-1)^{2016} - b(-1)^{2017} - 2(-1) +1 = a+b +3 = 1\]sehingga $a+b = -2.$

\item \textbf{Jawaban: 0.} Kita tinjau polinom $f(x) = 4x^2-6px -4q$ terlebih dahulu. Misalkan $f(x)$ dinyatakan dengan \[2(x-a)(2x-b) \equiv 4x^2-6px - 4q = f(x)\]dengan $b/2$ merupakan akar dari $f(x)$. Perhatikan bahwa \[2(x-a)(2x-b) = 2(2x^2-x(2a+b)+ab)\equiv 2(2x^2-3px-2q)\]sehingga kita dapatkan sistem persamaan
	\begin{gather}
	2a + b = 3p\label{nomor25.1}\\
	ab = -2q\label{nomor25.2}.
	\end{gather}
Kemudian selanjutnya, kita tinjau polinom $g(x) = 2x^2+2q$. Sama seperti sebelumnya, misalkan $g(x)$ dapat dinyatakan dengan $$2(x-a)(x-c) \equiv 2x^2+2q = g(x)$$ dengan $c$ merupakan akar dari polinom $g(x)$. Perhatikan bahwa $$2(x-a)(x-c) = 2(x^2-x(a+c)+ac) \equiv 2(x^2+q)$$ sehingga kita dapat sistem persamaan
	\begin{gather}
	a+c = 0\label{nomor25.3}\\
	ac = q\label{nomor25.4}
	\end{gather}
dari \eqref{nomor25.3} kita dapat $-a = c$, sehingga $a^2 = -q$. Subsitusikan $a^2=-q$ ke \eqref{nomor25.2}, didapat \[ab = 2a^2\Leftrightarrow b = 2a\quad\text{karena $a\neq 0$}\]selanjutnya, subsitusikan $b = 2a$ ke \eqref{nomor25.1} sehingga didapat \[p = \frac{4}{3}a.\]Dengan menggunakan semua hasil diatas, didapat \[9p^2 + 16q = 9\left(\frac{16}{9}a^2\right) + 16(-a^2) = 0.\]

\item

\item \textbf{Jawaban: $\boldsymbol{10p, p\in\mathbb{N}}$.} Misalkan $S_1$ dan $S_2$ menyatakan jumlah data berat badan siswa pada perhitungan pertama dan kedua berturut-turut. Maka \[S_1 = \underbrace{65+65+\ldots+65}_{\text{p kali}} + x\]dengan $x$ menyatakan jumlah data selain data yang salah dan $p$ menyatakan jumlah siswa yang datanya salah terbaca. Kemudian perhatikan bahwa karena \[S_2 = \underbrace{55+55+\ldots+55}_{\text{p kali}} + x\]maka didapat hubungan antara $S_1$ dan $S_2$ adalah \begin{equation}\label{nomor27.1}S_1 = S_2 + \underbrace{10+10+\ldots+10}_{\text{p kali}} = S_2 + 10p.\end{equation}Dengan meninjau rata-ratanya, dengan $n$ menyatakan banyak siswa \[\frac{S_1}{n} = 42\quad\text{dan}\quad\frac{S_2}{n} = 41\]perhatikan bahwa \[\frac{S_1}{n} - 1 = \frac{S_2}{n} = 41\]sehingga didapat \[\frac{S_1 - S_2}{n} = 1\Leftrightarrow S_1 - S_2 = n.\] Dari \eqref{nomor27.1} didapat $S_1 - S_2 = 10p$ sehingga \[n = 10p\quad\text{dengan sembarang $p\in\mathbb{N}$}\]

\item \textbf{Jawaban: 9/44.} Kita bagi menjadi beberapa kasus.
	\begin{enumerate}
	\item Terambil 3 bola kuning sekaligus:
		\par Banyak cara terambil 3 bola kuning sekaligus adalah \[\binom{5}{3} = 10\text{ cara,}\]sehingga peluangnya adalah \[P(A) = \frac{10}{\displaystyle \binom{5+7}{3}} = \frac{10}{220} = \frac{1}{22}\]
	\item Terambil 3 bola biru sekaligus:
		\par Banyak cara terambil 3 bola biru sekaligus adalah \[\binom{7}{3} = 35\text{ cara,}\]sehingga peluangnya adalah \[P(B) = \frac{35}{\displaystyle \binom{5+7}{3}} = \frac{35}{220} = \frac{7}{44}\]
	\end{enumerate}
Maka peluang terambil 3 bola berwarna sama sekaligus adalah \[P = P(A) + P(B) = \frac{1}{22} + \frac{7}{44} = \frac{9}{44}\]

\item \textbf{Jawaban: $\boldsymbol{-14\pm4\sqrt{15}}$.} Kita akan menggunakan persamaan \begin{equation}\label{nomor29.1}y_1 - y_p = m(x_1-x_p) \pm r\sqrt{m^2+1}\end{equation}dimana $(x_1,y_1),(x_p,y_p), r,$ dan $m$ adalah titik $(-1,-2)$, titik pusat lingkaran, jari-jari lingkaran dan gradien garis singgung yang melewati titik $(-1,-2)$ berturut-turut. Dari persamaan lingkaran pada soal, didapat \[x^2+y^2 - 2x - 10y + 21 = 0\Leftrightarrow (x-1)^2 + (y-5)^2 = 5\]sehingga didapat titik pusat lingkaran $(1,5)$ dan jari-jarinya $\sqrt{5}$. Dengan menggunakan persamaan \eqref{nomor29.1} diatas,
	\begin{align*}
	\text{\eqref{nomor29.1}} = -2 - 5 &= m(-1 -1)\pm \sqrt{5}\times\sqrt{m^2+1}\\
			-7 &= -2m \pm \sqrt{5m^2+5}\\
			(2m - 7)^2 &= 5m^2+5\\
			4m^2-28m+49 &= 5m^2+5\\
			\Leftrightarrow m^2+28m-44 &= 0
	\end{align*}
bisa dicek bahwa solusi dari persamaan kuadrat diatas adalah \[(m_1, m_2) = (4\sqrt{15}-14, -4\sqrt{15}-14)\]sehingga gradien garis singgungnya adalah $m = -14\pm4\sqrt{15}$

\item \textbf{Jawaban: 2.} Kita tinjau titik $A(2,3)$ terlebih dahulu \[T_1 \longrightarrow \begin{pmatrix}a & b\\0 & 1\end{pmatrix}\times\begin{pmatrix}2\\3\end{pmatrix} = \begin{pmatrix}2a + 3b\\3\end{pmatrix}\]kemudian\[T_2\longrightarrow \begin{pmatrix}0 & 1\\ -1 & 1\end{pmatrix}\times\begin{pmatrix}2a + 3b\\ 3\end{pmatrix} = \begin{pmatrix}3 \\ -2a-3b+3\end{pmatrix}\]sehingga didapat $(3, 3-2a-3b) = (3,4)$ sehingga \begin{equation}\label{nomor30.1}2a+3b = -1.\end{equation} Kemudian kita tinjau titik $B(-4,1)$, \[T_1 \longrightarrow \begin{pmatrix} a & b\\ 0 & 1\end{pmatrix}\times\begin{pmatrix}-4\\1\end{pmatrix} = \begin{pmatrix}b - 4a\\1\end{pmatrix}\]kemudian\[T_2 \longrightarrow \begin{pmatrix} 0&1\\-1&1\end{pmatrix}\times\begin{pmatrix}b-4a\\1\end{pmatrix} = \begin{pmatrix}1 \\ 4a -b +1\end{pmatrix}\]sehingga didapat $(1,6) = (1, 4a-b+1)$ sehingga \begin{equation}\label{nomor30.2}4a-b = 5.\end{equation}Dengan menyelesaikan sistem persamaan \eqref{nomor30.1} dan \eqref{nomor30.2} akan didapat \[(a,b) = (1,-1)\] sehingga nilai \[3a+b = 3 + (-1) = 2.\]

\item \textbf{Jawaban: 109.} Misalkan 5 bilangan tersebut adalah $x_1,\ x_2,\ x_3,\ x_4,\ x_5$ dengan $x_i > x_j$ untuk setiap $i>j$. Sehingga didapat sistem persamaan
	\begin{gather}
	x_1 + x_2 + x_3 + x_4 = 211\label{nomor31.1}\\
	x_1 + x_2 + x_3 + x_5 = 214\label{nomor31.2}\\
	x_1 + x_2 + x_4 + x_5 = 216\label{nomor31.3}\\
	x_1 + x_3 + x_4 + x_5 = 221\label{nomor31.4}\\
	x_2 + x_3 + x_4 + x_5 = 222\label{nomor31.5}
	\end{gather}
jika semua persamaan diatas dijumlahkan akan didapat \[4(x_1 + x_2 + x_3 + x_4 + x_5) = 211 + 214 + 216 + 221 + 222 = 1084\]sehingga \begin{equation}\label{nomor31.6}x_1+x_2+x_3+x_4+x_5 = 271.\end{equation} Nilai $x_5$ bisa didapat dari \eqref{nomor31.6} - \eqref{nomor31.1}, sedangkan nilai $x_1$ bisa didapat dari \eqref{nomor31.6} - \eqref{nomor31.5} sehingga didapat $(x_1, x_5) = (49, 60)$. Maka nilai dari $x_1 + x_5 = 49 + 60 = 109$

\item \textbf{Jawaban: 2017/2018.} Perhatikan bahwa \[U_n = \frac{1}{n(n+1)} = \frac{1}{n} - \frac{1}{n+1}\]sehingga didapat
\begin{align*}
U_1 +U_2 +\ldots +U_{2017} &= \left(\frac{1}{1} - \frac{1}{2}\right) + \left(\frac{1}{2} - \frac{1}{3}\right) + \ldots + \left(\frac{1}{2017}-\frac{1}{2018}\right)\\
				      &= 1-\bigl(2017/2018\bigr)\\
				      &= \frac{2017}{2018}
\end{align*}

\item \emph{Gambar soal tidak jelas.}

\item

\item \textbf{Jawaban: 9.} Misalkan $$x_1,x_2,x_3,\ldots,x_6$$ menyatakan semua nilai tes matematika siswa tersebut, dengan $x_i\geq x_j$ untuk setiap $i>j$. Median dari ke-enam data tersebut adalah $$\text{Me} = \frac{x_3+x_4}{2},$$ karena rata-ratanya 6, maka $$x_1 + x_2 + \ldots +x_6 = 39$$karena kita ingin memaksimalkan $x_3+x_4$, maka $x_1$ dan $x_2$ harus minimum. Kita tetapkan $x_1 = x_2 = 1$, sehingga $$x_3 + x_4+x_5+x_6 = 37.$$Sekarang, jika $x_3 = 10$ maka $x_4=x_5=x_6 =10,$ tetapi ini tidak mungkin karena jika demikian, maka$$x_3+x_4+x_5+x_6 = 40 > 37$$ sehingga haruslah $x_3 = 9.$ Jika $x_3 = 9$, ada dua kemungkinan nilai $x_4$, yaitu $x_4 = 9$ atau $x_4 = 10$. Saat $x_4 = 10$, berarti $x_5 = x_6 = 10,$ tetapi sekali lagi ini tidak mungkin karena $$x_3+x_4+x_5+x_6 = 39 > 37$$sehingga haruslah $x_4 = 9.$ Seperti sebelumnya, ada dua kemungkinan nilai $x_5$, yaitu $x_5 = 9$ atau $x_5 = 10$, sama seperti alasan sebelumnya, dapat dicek bahwa $x_5 = 10$ tidak memenuhi sehingga haruslah $x_5 = 9$. Sekarang, karena $$x_3 + x_4 + x_5 = 27$$maka haruslah $x_6 = 10.$ Dengan demikian, median terbesar yang mungkin dari data tersebut adalah $$\text{Me} = \frac{9+9}{2} = 9$$

\end{enumerate}
