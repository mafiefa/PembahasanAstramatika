\chapter{Kumpulan Soal Mandiri}
\hrulefill

\section{Soal Mandiri - Astramatika XXIV}
\begin{enumerate}

\item Nilai $x$ yang menyebabkan pernyataan:
	\begin{center}
	\emph{``Jika $x^2+2x = 24$ maka $x^2+4x>17$''}
	\end{center}
bernilai salah adalah \ldots

\item Jika $p$ dan $q$ adalah akar-akar persamaan $x^2-3x+4 = 0$ maka nilai $$(p^2-p+4)(q^2+q+4)$$ adalah \ldots

\item Jika $\{x\in\mathbb{R} \mid x<p \textrm{ atau } x>q\}$ merupakan himpunan penyelesaian dan memenuhi $1<a<2$ dari pertidaksamaan $${\displaystyle \frac{ax^2-3x+5}{3+8x-3x^2}\leq 0},$$maka nilai $3p+2q$ adalah \ldots

\item Absis titik balik grafik fungsi $f(x) = ax^2 + (a+6)x + 8$ adalah $-a$ dengan $a$ bilangan bulat. Nilai maksimum fungsi tersebut adalah \ldots

\item Garis $y=2x+k$ memotong parabola $y=x^2-x+3$ di titik $(a,b)$ dan $(c,d)$. Jika $a^2+c^2 = 7$ maka nilai $k$ adalah \ldots

\item Jika $${\displaystyle \sqrt{\frac{b}{a}\sqrt{\frac{a}{b}\sqrt{\frac{b}{a}\sqrt{\frac{a}{b}\cdots}}}}} = a^xb^{-x},$$ maka nilai $x$ adalah \ldots

\item Jika $a^{2x} + a^{-2x} = 3$, maka nilai dari $a^x - a^{-x}$ adalah \ldots

\item Diketahui $\log_2 56 = 7a, \log_7 2 = b, \log_4 1024 = c$ dan $${\displaystyle \left(\frac{1}{b} + \frac{50}{c}\right)^2 = d},$$ untuk $d=0$. Maka nilai $a$ adalah \ldots

\item Selisih dua bilangan positif adalah 5. Jumlah kuadratnya 2.100 kurangnya dari kuadrat jumlahnya. Jumlah kedua bilangan tersebut adalah \ldots

\item Jika $x$ adalah bilangan bulat negatif dan $3b+3x = 2a, 2x+a = b, 4a-5b = c$ dimana $a,b,$ dan $c$ adalah bilangan real. Nilai minimum dari $a+b+c$ adalah \ldots

\item Jika titik $(3,2), (11,0),$ dan $(1,k)$ terletak pada satu garis lurus, maka nilai $k$ adalah \ldots

\item Pada daerah $3x+2y \geq 24, 2x+2y\geq 20, x,y\geq 0$. Fungsi objektif $$f(x,y) = ax+4y$$ dimana $a$ adalah bilangan bulat, mencapai nilai minimum di titik $(4,6)$ jika nilai $a = \ldots$

\item Daerah definisi fungsi $$f(x) = \sqrt{\frac{x^2-x+2}{x-2}}$$ adalah \ldots

\item Diketahui $f^{-1}(8x-10) = 6x-2$ dan $(f^{-1} \circ f)(10) = p^2+4p-20$ maka hasil kali akar-akar dari $(f^{-1} \circ f)(10) = p^2+4p-20$ adalah \ldots

\item Hasil dari $\sin(\ang{2550}) + \tan(\ang{2025})$ adalah \ldots

\item Jika $${\displaystyle \sin\left(\frac{1}{2}\theta\right) = \frac{1}{5}\sqrt{5}},$$ maka $\sin(2\theta)$ adalah \ldots

\item Hasil dari $${\displaystyle \cos\left(\frac{\pi}{8}\right) + \cos\left(\frac{3\pi}{8}\right) + \cos\left(\frac{5\pi}{8}\right) + \cos\left(\frac{7\pi}{8}\right)}$$ adalah \ldots

\item Hasil dari $${\displaystyle \lim_{x\to1}\frac{(2x-3\sqrt{x}+1)(\sqrt{x}-1)}{(x-1)^2}}$$ adalah \ldots

\item Turunan pertama fungsi $f(x) = (\sin x + \cos x)^4$ adalah $f'(x)$. Nilai $f'(x)$ adalah \ldots

\item Turunan pertama $$y = -\cos(-\cos(-\cos\cdots(-\cos x)\cdots))$$ pada $x = \frac{\pi}{2}$ adalah \ldots

\item Nilai dari $$\int \sin^4(3x)\cos(3x) \dx$$ adalah \ldots

\item Daerah $D$ terletak antara garis $y=2x$, garis $x=3$, dan garis $y=0$. Volume benda putar yang terjadi jika $D$ diputar mengelilingi garis $x=3$ adalah \ldots

\item Besar sudut antara vektor $3\hat{i} + 4\hat{j} + 2\hat{k}$ dengan vektor $2\hat{i} - 3\hat{j} + 3\hat{k}$ adalah \ldots

\item Sisa pembagian $ax^{2016} - bx^{2017} - 2x+1$ oleh $x^2-1$ adalah $x+2$. Nilai $a+b$ adalah \ldots

\item Diketahui $4x^2-6px-4q$ dan $2x^2+2q$ mempunyai faktor yang sama yaitu $(x-a)$, dengan $p,q,$ dan $a$ merupakan konstanta bukan nol. Nilai dari $9p^2 + 16q$ adalah \ldots

\item Bidang $U$ dan $V$ berpotongan pada garis $g$ dengan sudut $\theta$ di titik $P$ pada bidang $U$, jarak titik $P$ dengan bidang $V$ adalah 1. Jika $\tan \theta = \nicefrac{3}{4}$, maka jarak titik $P$ dengan garis $g$ adalah \ldots

\item Rata-rata berat badan sekelompok siswa dihitung 2 kali. Pada perhitungan pertama diperoleh rata-rata 42 dan pada perhitungan kedua diperoleh rata-rata 41. Perbedaan perhitungan disebabkan kesalahan membaca data yaitu 55 dibaca 65. Banyaknya siswa pada kelompok tersebut adalah \ldots

\item Sebuah kotak berisi 5 bola kuning dan 7 bola biru. Jika diambil 3 bola sekaligus, peluang terambilnya semua bola berwarna sama adalah \ldots

\item Kedua garis lurus yang ditarik dari titik $(-1,-2)$ dan menyinggung lingkaran dengan persamaan $x^2+y^2-2x-10y+21=0$ mempunyai gradien \ldots

\item $A'(3,4)$ dan $B'(1,6)$ adalah bayangan dari $A(2,3)$ dan $B(-4,1)$ oleh transformasi $T_1 = \begin{bmatrix} a & b\\ 0 & 1\end{bmatrix}$ diteruskan $T_2 = \begin{bmatrix} 0 & 1 \\ -1 & 1\end{bmatrix}$. Nilai $3a + b$ adalah \ldots

\item Untuk 5 bilangan positif yang berbeda, jumlah setiap 4 bilangan adalah 221, 216, 214, 222, dan 211. Tentukanlah hasil dari penjumlahan bilangan terbesar dengan yang terkecil!

\item Suku ke-$n$ suatu barisan adalah $$U_n = \frac{1}{n^2+n}.$$ Tentukan nilai dari $U_1 + U_2 + U_3 +\ldots + U_{2017}$.

\item Titik $B$ dan $C$ terletak pada kurva $y=x^2$ dan $BC$ sejajar dengan sumbu-$x$. Hitunglah luas maksimum segitiga $ABC$. \par \emph{Gambarnya tidak jelas.}

\item Tentukan volume benda putar yang terbentuk dari daerah yang dibatasi oleh kurva $y=-x^2\sqrt{3},$ sumbu-$x$, didalam lingkaran $x^2+y^2=4$ diputar mengelilingi sumbu $x$.

\item Nilai semua tes matematika dinyatakan dengan bilangan bulat dari 0 sampai 10. Tentukanlah median terbesar yang mungkin bagi siswa yang memiliki rata-rata 6,5 dari enam kali tes!

\end{enumerate}